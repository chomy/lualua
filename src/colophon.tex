\section*{編集後記}
``nginxでLuaLuaする''をお送りします。
実は、今年(2016年)の2月にタイトルだけ思いついていて、認証系の何かを
書こうと思っていましたが、実は6月までちゃんと決まっていませんでした。
6月にCISSPのセミナーに出て、AzureのAD認証の話を聞いたのがきっかけで
多要素認証の話を書くことにしました。

5月のパスワード定期変更のニュースは衝撃でした。しかし報道では
他要素認証が必須という肝心な部分が抜け落ちていた、または書いてあっても
最後の方にちょこっと書いてあるくらいの扱いでしたので、
このニュースを根拠に、弱いパスワードであるにもかかわらず、変更せず
ドヤ顔で使い続ける人が増えそうで、非常に心配です。


多要素認証といいながら、TOTPだけしか実装していません。パスワードの
認証機能も入れたかったのですが、時間切れでした。
サーバ側のパスワードの保存方法は、語り尽くされていますので
みなさんへの課題としておきましょう。
断じて、DBに平文で保存してはいけません。ソルト、ハッシュ、ストレッチです。
この辺りのことも含めて、後日加筆するかもしれません。
この原稿は、GitHubで管理されていますので、下にあるGitHubリポジトリを
watchすると更新時にお知らせが届きます。

紙でこの同人誌を入手された方は、下のGitHubリポジトリに
PDF版も入手可能です。また電子版で入手したけれども紙も欲しいという方も
面付け済みのPDFをGitHubにアップロードしていますので、短辺とじ両面印刷
でコピー本を作成することができます。

最後に、この同人誌はDebian/GNU Linux、\TeX Live2016、
psutils、git、GNU Make、vimといった、オープンソースソフトウェアを使って作成されました。フォントはIPAexフォントをPDFに埋め込んでいます。
ありがとうございました。

\begin{flushright}
2016年8月13日 Keisuke Nakao (@jm6xxu) 
\end{flushright}
\subsection*{参考文献}
\begin{itemize}

\item ``Microsoft Password Guidance'' \\
https://www.microsoft.com/en-us/research/publication/password-guidance/
\item ``DRAFT NIST Special Publication 800-63B Digital Authentication Guideline''\\
https://pages.nist.gov/800-63-3/sp800-63b.html
\item ``Key Uri Format''\\
https://github.com/google/google-authenticator/wiki/Key-Uri-Format
\end{itemize}
%\clearpage
%\mbox{}
%\clearpage
\mbox{}\\
\vspace{48em}\\
この作品はクリエイティブ・コモンズ・ライセンス 表示 - 継承 2.1 日本 の下に提供されています。このライセンスのコピーを見るためには、http://creativecommons.org/licenses/by-sa/2.1/jp/ をご覧ください。
