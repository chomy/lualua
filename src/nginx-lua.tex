\chapter*{lua-nginx-module}

lua-nginx-module は、Webサーバであるnginxを、軽量プログラミング言語
Luaで拡張できるというモジュールです。
以下のようにLuaで記述可能なフックがいくつも用意されていますので、様々な拡張が可能です。

\begin{figure}
\scalebox{.3}{\includegraphics{order.ps}}
\caption{Luaで記述可能なフックと呼び出される順番}
\end{figure}

よく使われるのは、init、rewrite、access、content、logでしょうか。
ちょっと見ないうちに、ssl絡みやfilter絡みのフックが増えています。
詳細は、openrestyのlua-nginx-moduleのGitHubリポジトリを参照してください。

\section*{Hello World}
では、さっそくやってみましょう。
nginx.confのserverブロックに以下の内容を追加します。
\lstset{
  basicstyle={\ttfamily\small},%
  identifierstyle={\small},%
  commentstyle={\small\itshape},%
  keywordstyle={\small\bfseries},%
  ndkeywordstyle={\small},%
  stringstyle={\small\ttfamily},
  frame={tb},
  breaklines=true,
  columns=[l]{fullflexible},%
  numbers=left,%
  xrightmargin=0zw,%
  xleftmargin=3zw,%
  numberstyle={\scriptsize},%
  stepnumber=1,
  numbersep=1zw,%
  lineskip=-0.5ex%
}
\begin{lstlisting}[caption=hello world,label=list:hello-world]
server{
         .... (snip)

  location /hello{
    default_type text/plain;
    content_by_lua 'ngx.say("Hello World")';
  }
\end{lstlisting}

変更したら、nginxを再起動させて、/helloにアクセスしてみましょう。
自分のPCにnginxをインストールして試している人は、http://localhost/hello
にブラウザでアクセスして、Hello Worldと表示されたら成功です。

次にGETリクエストを取得してみましょう。
\begin{lstlisting}[caption=GETパラメータの取得,label=list:get]
server{
         .... (snip)

  location /hello{
    default_type text/plain;
    content_by_lua '
      local args = ngx.req.get_uri_args()
      ngx.say("Hello " .. args.name)';
    }
\end{lstlisting}
http://localhost/hello?name=chomy にアクセスして、Hello chomyと表示される
でしょう。最後はPOSTです。
\begin{lstlisting}[caption=POSTパラメータの取得,label=list:get]
server{
         .... (snip)

  location /hello{
    default_type text/plain;
    content_by_lua '
      ngx.req.read_body()
      local args = ngx.req.get_post_args()
      ngx.say("Hello " .. args.name)';
    }
\end{lstlisting}

その他、ヘッダの読み書きは、
\begin{lstlisting}[caption=ヘッダの読み書き,label=list:hello-world]
ngx.header.content_type = 'text/plain';
ngx.header["X-My-Header"] = 'blah blah';
ngx.header['Set-Cookie'] = {'a=32; path=/', 'b=4; path=/'}
\end{lstlisting}
%
ngx.header変数でアクセスできます。クライアントへのステータスコードの
返却は、ngx.exit 関数で行います。

\section*{使用可能なライブラリ}
nginxの中でLua5.1インタープリタが動いているので、OpenResty以外のものも
Lua5.1で動くライブラリは使用可能です。
インストールされているライブラリのPATHは、lua\_package\_path で
共有ライブラリのPATHは、lua\_package\_cpath で指定します。

