\chapter*{はじめに}

2016年の5月に、IT分野で大事件が起こりました。
これまでユーザを悩ませ続けてきたパスワードの定期変更が無意味である事が
マイクロソフトによって公開され\cite{MS}、またNISTもそれに続きました。
私はこれに納得できませんでした。
パスワードの定期変更は、ユーザだけでなくサービスを提供する側の落ち度
でパスワードが漏洩した場合の対策だったはずでした。

問題の文書``Microsoft Password Guidance''を入手して読んでみると
衝撃的なことが書いてありました。曰く、
\begin{quote}
ユーザにパスワードの定期変更や、長くて大文字小文字数字を含むパスワードを
要求しても、クソみたいに弱いパスワードしか使わない。
\end{quote}
曰く、
\begin{quote}
そんなもん意味ねーから、多要素認証使おうぜ。
\end{quote}

はい。これで納得がいきました。パスワードはオワコン、これからは多要素認証の
時代だぜ!という内容でした。

と、いうことで、今回はWebサーバであるnginxで、Google Authenticator を
使った多要素認証モジュールを作ってみましょう。
