\chapter*{Time based One Time Password}

さて、認証の話に戻りましょう。これから多要素認証システムを
nginxに組み込むわけですが、パスワード以外の何を使って認証するか
決めなければなりません。よくあるのは、指紋や静脈と行った生体認証
ですが、これはスキャナが別途必要なので現実的ではありません。
次はハードウェアワンタイムトークンなのですが、Yubikeyの登場で
ずいぶん入手性が良くなったとはいえ、まだ敷居が高いようです。
今はみんなスマートフォンを持っているので、スマートフォンをトークンに
する手があります。今回はGoogle Authenticator や、IIJ SmartKey が使っている
TOTP (Time based One Time Password)を採用しましょう。

\section*{TOTP}
TOTPはRFC6238で規定されています。ワンタイムパスワードの生成は簡単です。
\begin{enumerate}
\item 鍵となる数値を$secret$とする
\item パスワードの有効期限を30秒とし、epochからの秒数を30で割って、小数点以下を切り捨てた整数値を$step$とする
\item この$secret$を鍵として、stepをコンテンツとするHMAC-SHA1を計算
\item 計算したHMAC-SHA1の下位4bitを$offset$とする
\item HMAC-SHA1のoffsetバイトから4byte取り出し、0x7FFFFFFFとANDをとる
\item 最後に必要な桁数を下位の桁から取り出す
\end{enumerate}

注意してほしいのは、$secret$も$step$もビッグエンディアンで使用する
ということです。ここでずいぶんハマりました。

TOTPではユーザとサービス・プロバイダでHMAC-SHA1を計算する時に必要な鍵を
共有します。この共有鍵と時刻をパラメータにした同じアルゴリズムで
計算した値を比較することでユーザの認証を行います。

この方式の素晴らしい点は、ログイン時に鍵が通信経路に乗らないことです。
また、ログインに使用するトークンは30秒毎に変化しますので、
仮に中間者攻撃を受けてトークンが漏洩しても、そのトークンが使えるのは30秒間
だけです。
また、トークンから鍵を復号するにはものすごく莫大な計算量が必要で
現実的ではありません。

さて、実装時にはLuaでトークンを計算するのですが、どうやらnginxに
組み込まれているLua5.1は、ビット演算をサポートしていないようです。
よって、泣く泣くC++で関数を書いてFFI(Foreign Function Interface)を使って
呼び出すことにした。トークンを計算するOpenSSLを使ったC++のコードを
Listing \ref{list:gettoken} に示します。

\begin{lstlisting}[caption=トークンの計算(C++),label=list:gettoken]
uint32_t get_token(const unsigned char *key,const int keylen, 
             	   const unsigned char *step, const uint16_t digits)
{
  unsigned char hash[SHA_DIGEST_LENGTH];
  unsigned int len;

  HMAC(EVP_sha1(), key, keylen, 
       step, 8, hash, &len);

  uint32_t binary = ((hash[offset]&0x7F)<< 24)
     | ((hash[offset+1]&0xFF)<< 16)
     | ((hash[offset+2]&0xFF)<<8)
     | ((hash[offset+3]&0xFF));

  return binary%(uint64_t)powl(10,digits);
}
\end{lstlisting}

Luaから呼ばれる関数は、extern "C"宣言が必要です。
C++11の機能を使っています。ここでは使っていませんが
右辺値参照がすばらしいので、もうC++99には戻れません。
g++6.1はデフォルトでC++14とのことですので、
もうC++99は忘れても良いでしょう。

\begin{lstlisting}[caption=呼び出し用の関数,label=list:caller]
typedef std::vector<uint8_t> byte_array;
typedef std::array<unsigned char, 8> step_t;

extern "C" void get_token(const char *key, char *result, uint16_t len)
{
        step_t step;
        byte_array bKey;
        
		{
        	uint64_t t = htobe64(ticks/30);
        	memcpy(step.data(), &t, step.size());
		}

        hex2byte(key, &bKey);

        uint32_t tkn = get_token(bKey.data(), 10, step.data(), len);
        stringstream s;
        s<<setfill('0')<< setw(6)<< tkn;
        strncpy(result, s.str().c_str(), len);
        result[len-1] = '\0';
}
\end{lstlisting}

最後にコンパイルして、共有ライブラリを作成します。
こんな感じになるでしょう。

\begin{lstlisting}[caption=共有ライブラリの作成,label=list:sharedlib]
$ g++ -fPIC -shared --std=c++11 `pkg-config --cflags --libs openssl` -o libtotp.so totp.cc
\end{lstlisting}
これで、トークンを計算するlibtotp.soができました。
